% !TEX root = Thesis.tex
\section{Offshoring in literature}
Offshoring has been widely studied in the past decades. There are two major branches of research: the first describes reality through statistics or case studies (e.g. \cite{Rottman.2008} and \cite{Pedersen.2013}) while the second branch designs trade models to explain the discovered correlations (e.g. \cite{Antras.2004}, \cite{Grossman.2008} and \cite{Helpman.1999}).

This wealth of existing knowledge has been used for the following section, where the relevant terms of the subject are defined first. Furthermore, a brief history of offshoring is given before describing offshoring in the USA first, then offshoring in Germany.

\subsection{Definition and Terms}
In existing literature, there is no single definition of the term offshoring nor a precise delimitation to the term outsourcing. Both apply to organizational decisions in companies. 

According to \cite[pp. 1f]{Specht.2007}, outsourcing is buying services from other companies. Offshoring is defined as a special form of outsourcing, in which the service is bought from a foreign company. \cite[p. 2]{Alebrand.2013} defines outsourcing and offshoring as mutually exclusive: outsourcing is the provision of services by external companies, offshoring is the internal execution of tasks in a foreign country.

These contrasting definitions may serve as an example for the lack of distinct terms in this field of research. Nevertheless all the definitions agree that outsourcing pertains to external service provision and offshoring refers to service provision in a foreign country. This Bachelor's thesis will use the following definition of the term offshoring by \cite[p. 321]{Andersson.2016}:

\begin{quote}
	``Offshoring [is the] disintegration of the firms’ production processes across national borders[...]''
\end{quote} 

This means, offshoring is not only a description for the state of an organization, but also the process to relocate business processes.

The term outsourcing is derived from ``Outside Resource Using''(\cite[p. 46]{Specht.2007b}). It is acquiring intermediate inputs from external businesses (see ibid.).

Therefore, the terms offshoring and outsourcing do not have a direct relation; both terms are independent and describe different possibilities of entrepreneurial organization. In figure \ref{fig:DefTerms}, the delimitation between outsourcing and offshoring is clearly shown. A company can choose to offshore, outsource or both; every single possibility has its own term. Offshoring, in the context of this thesis, means foreign outsourcing and foreign direct investment, unless otherwise specified.

\begin{figure}[htb]
	\centering
	\includegraphics[width=0.7\textwidth]{Pictures/Terms_definition}
	\caption{Definition of terms, based on \cite[pp. 552f]{Antras.2004}}
	\label{fig:DefTerms}
\end{figure}

\subsection{History of Offshoring}
%Enabling Technologies
%
Globalization, and offshoring as part of the development, had its early beginnings in the 1970s and gained traction once the Iron Curtain had fallen in 1990.

\subsection{Offshoring in the USA}

\subsubsection{Prevalence}

\subsubsection{Offshoring functions}

\subsection{Offshoring in Germany}

\subsubsection{Prevalence}

\subsubsection{Offshored functions}

\subsection{Significant differences between Germany and the USA}

\subsubsection{Difference 1}

\subsubsection{Difference 2}
