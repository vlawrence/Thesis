% !TEX root = Thesis.tex
\section{Offshoring in Literature}
Offshoring has been widely studied in the past decades. There are two major branches of research: the first describes reality through statistics or case studies (e.g. \cite{Rottman.2008} and \cite{Pedersen.2013}) while the second branch designs trade models to explain the discovered correlations (e.g. \cite{Antras.2004}, \cite{Grossman.2008} and \cite{Helpman.1999}).

This wealth of existing knowledge has been used for the following section, where the relevant terms of the subject are defined first. Furthermore, a brief history of offshoring is given before describing offshoring in the USA first, then offshoring in Germany.

\subsection{Definition and Terms}
In existing literature, there is no single definition of the term offshoring nor one precise delimitation to the term outsourcing. Both terms refer to sourcing decisions in companies. 

For example, according to \cite[pp. 1f]{Knolmayer.2007}, outsourcing is buying services from other companies. Offshoring is defined as a special form of outsourcing, in which the service is bought from a foreign company. On the other hand, \cite[p. 2]{Alebrand.2013} defines outsourcing and offshoring as mutually exclusive: outsourcing is the provision of services by external companies, offshoring is the internal execution of tasks in a foreign country.

These contrasting definitions may serve as an example for the lack of distinct terms in this field of research. Nevertheless all the definitions agree that outsourcing pertains to external service provision and offshoring refers to service provision in a foreign country. This Bachelor's thesis will use the following definition of the term offshoring by \cite[p. 321]{Andersson.2016}:

\begin{quote}
	``Offshoring [is the] disintegration of the firms’ production processes across national borders[...]''
\end{quote} 

This means, offshoring is not only a description for the state of an organization, but also the process to relocate business processes.

The term outsourcing is derived from ``Outside Resource Using''(\cite[p. 46]{Specht.2007b}). It is acquiring intermediate inputs from external businesses (\cite[p. 46]{Specht.2007b}).

Therefore, the terms offshoring and outsourcing do not have a direct relation; both terms are independent and describe different possibilities of entrepreneurial organization. In figure \ref{fig:DefTerms}, the delimitation between outsourcing and offshoring is clearly shown. A company can choose to offshore, outsource or both; every single possibility has its own term. Offshoring, in the context of this thesis, means foreign outsourcing and \ac{FDI}, unless otherwise specified.

\begin{figure}[htb]
	\centering
	\includegraphics[width=0.7\textwidth]{Pictures/Terms_definition}
	\caption{Definition of terms, based on \cite[pp. 552f]{Antras.2004}}
	\label{fig:DefTerms}
\end{figure}

Some publications add a geographical dimension to their definition of offshoring. \cite{Jahns.2007} state that there must be shores between the customer and the supplier in order to call the transaction offshoring, otherwise the correct term would be Nearshoring. \cite{Cappallo.2006} postulate a need for a relatively high distance ("relativ hohe räumliche Distanz", p. 487) between the partners. \cite{Dressler.2007} defines offshoring only as transaction between partners on different continents. Given the ambiguous and arbitrary nature of the distinction between offshoring and nearshoring, this thesis refrains from using the term nearshoring. All provision of services outside the captive country of a company are called offshoring.

\subsection{Factors for the Development of Offshoring}

Globalization, and offshoring as part of the development, had its early beginnings in the 1970s and gained traction once the Iron Curtain had fallen in 1989 (\cite[p. 1]{Sachs.1995}). This section describes the various factors that enabled the development of offshoring to the point it is today.

\paragraph{Political and Historical Developments}
After the end of \ac{WWII}, countries belonged to one of three distinct sectors of the world: the capitalist western countries, communist eastern companies or developing countries that sought a way to not get crushed between the two super powers and proclaimed state-led industrialization, a third way between capitalism and communism. (\cite[pp. 12f]{Sachs.1995})

With the majority of world population in countries without market-based economic mechanisms in place and most of the currencies not freely convertible, international trade was basically nonexistent in the post-war world. While western countries systematically restored their trade relations, developing countries were much slower to open their economic systems to international trade. By 1994, most countries had opened their trade policies through removing trade barriers, ensuring the free convertibility of their currencies and disestablishing state monopolies. (\cite[pp. 12-25]{Sachs.1995})

In the last twenty years, global trade relations have only increased. Trade agreements and organizations such as the \ac{WTO}\footnote{For further information please see the website of the WTO, \url{https://www.wto.org/}, visited on 05. \nolinebreak August 2016}, the \ac{NAFTA}\footnote{Further information: \url{https://ustr.gov/trade-agreements/free-trade-agreements/north-american-free-trade-agreement-nafta}, visited on 05. August 2016} or  the \ac{EEC}\footnote{Established 1957 with the Treaty of Rome \url{http://eur-lex.europa.eu/legal-content/EN/TXT/?uri=URISERV:xy0023}, visited on 05. August 2016} (a predecessor of the European Union) further facilitated global trade and created a stable environment for long-term business agreements across borders. 

\paragraph{Information and Communication Technology}
Innovations in \ac{ICT} have been paramount in enabling offshoring. Beginning with the invention of the first computer in 1941, the rapid development of computing power, data storage and particularly data transmission removed the need for local completion of tasks. The Internet necessitated a quick standardization and modernization of communication systems on a global scale -- the prerequisite for offshoring. (\cite[pp. 9f]{Hutzschenreuter.2007} and \cite[p. 93]{Jahns.2007})

\paragraph{Organizational Factors}
In order to efficiently offshore tasks or processes, the work has to be well-defined and standardized. In this way, economies of scale can fully be utilized and completion of work can be managed across multiple involved companies or subsidiaries. (\cite[p. 11]{Hutzschenreuter.2007})

The aforementioned developments in \ac{ICT} remove the need for local presence of the service provider (Uno-Actu-Principle) for most services. Digitalization enables organizations to detach tasks from specific locations. In a first step, those tasks are centralized and standardized. The second step is often offshoring the tasks. (\cite[pp. 12f]{Hutzschenreuter.2007})

\paragraph{Macroeconomic and Socio-Demographic Factors}
\ac{ICT} developments, organizational and political factors are enablers for offshoring, but the main driver for offshoring decisions in companies are the differences in salaries, taxes, and interest rates between industrialized and developing countries that result in cost arbitrage. In \cite[p. 89]{Jahns.2007}, the example of engineering wages in 2000 is given: while German and American engineers earned \$31 and \$36 per hour, an Indian engineer made only \$6.5\footnote{Figures are given with a decimal point.} per hour\footnote{The authors refer to United Nations Secretary and Industry Labor Office (2002) as source of these wages, which could not be verified at the time of writing this thesis.}. It is obvious that companies want to use this disparity to their advantage.

In addition to the wage differences, socio-demographic factors such as education, motivation and age distribution in developing countries influence offshoring supply. High social prestige connected to working for large western companies contributes to a higher ratio of academics that apply for offshoring related jobs and motivates employees. Thus, the quality of work is often very good and may be better than in the original country. (\cite[p. 93]{Jahns.2007})

In the following sections, those factors for both the USA and Germany will be examined. Furthermore, a quantification of offshoring will be attempted in order to provide a basis for the direct comparison of both countries in section \ref{sec:DifferencesUSGER}.

\subsection{Offshoring in the USA}
\label{sec:OffshoringUS}

\paragraph{Political and Historical Developments}

\paragraph{Offshoring Quantified}
\begin{quote}
	``The United States is the world's largest direct investor[...]'' \linebreak(\cite[p. 3]{Kozlow.2006})
\end{quote}

Offshoring originated in the USA in the early 1990s (\cite[p. 389]{Pisani.2016}). Even earlier, U.S. companies pioneered in foreign investment, e.g. by establishing production sites abroad (\cite[p. 5]{Kozlow.2006}). Looking at the past 25 years, imports of services and especially imports of \ac{ICT} have grown exponentially. In figure \ref{fig:US_ICT_Import}], import volume for ICT services is shown. Short of a small decline in 2002, which can be explained with the burst of the Dotcom Bubble in the same year, volumes have consistently grown and tripled from \$12 bn in 1999 to \$36 bn in 2015. Since 2010, the growth has slowed down considerably. It remains to be seen if this trend persists or if it is just a small break in further growth.

It is difficult to accurately measure how many jobs have been shifted to countries outside of the U.S.. Often, one company may outsource to a different company in the U.S., which in turn could use a subcontractor in a different country. In this scenario, no company has actively shifted jobs abroad, but there is still an impact on the employment market. An estimation by \ac{BEA} of job losses due to offshoring was 195 000 jobs per year from 1999 to 2001, which is only 1.5\% of the 13 million jobs that were lost overall per year. (\cite[pp. 14ff]{Kozlow.2006})



\bildcite{US_Import_ICT_Services}{U.S. import of \acs{ICT} services}{fig:US_ICT_Import}{Data source: \url{www.bea.gov/newsreleases/international/trade/trad_time_series.xls}, visited on 10. August 2016}{0.7}



\subsection{Offshoring in Germany}
\label{sec:OffshoringGER}

\paragraph{Political and Historical Developments}
In the aftermath of \ac{WWII}, German economy was devastated. Vast areas of the country were destroyed by allied bombs, including cities and production plants. The country was devided into four military occupation zones, one of which would become the soviet-influenced \ac{GDR} in 1949.

In \ac{GDR}, the Soviet Union undertook an extensive industrial dismantling, while similar plans had not been executed in the allied occupation zones. Still, reconstruction of German economy progressed very slowly. Facing the threat of communist ideology spreading in Europe, Secretary of State Marshall established the ``Marshall Plan'', which allowed participating countries to receive U.S. goods and raw materials while paying in their local currency. This was the foundation for a rapid growth of West German economy between 1950 and 1960, the so-called `Wirtschaftswunder'. (\cite{Kimmel.2005})

On an international level, \ac{FRG} had quickly become a valued trading partner and important exporter of industrial machines, automobiles and %CITATION needed


\paragraph{Offshoring Quantified}
According to \cite[p. 70]{Eickelpasch.2015}, only 9.3 \% of business services have been imported in 2010\footnote{The author cites input-output tables of  Statistisches Bundesamt and calculations of DIW Berlin.}. This may seem like a very low number, even though it is expected that fewer German companies offshore, compared to the USA. However, Eickelpasch only accounts for Foreign Outsourcing as his definition of offshoring does not include \acp{FDI}(\cite[p. 56]{Eickelpasch.2015}). This information is therefore not sufficient to draw any conclusions concerning offshoring in Germany.

Further insights into the prevalence of offshoring in Germany can be found in a survey that has been conducted by German Statistisches Bundesamt in 2008. For this survey, 9361 manufacturing and service companies answered a questionnaire focusing on drivers, scope and results of offshoring on firm-level (\cite[p. 7]{StatistischesBundesamt.2008}). Of the polled service companies\footnote{The survey is on shifting business activities abroad, so it includes production abroad. This thesis focuses on offshoring services, so only results of service companies are included.}, 15.4\% had offshored one or multiple corporate functions until 2006, and 10.7\% planned to do so in the time span 2007 - 2009. The percentage of companies that offshore grows with the number of employees. (\cite[p. 11]{StatistischesBundesamt.2008})

Regarding cooperation partners, the survey found that 81.4\% of service companies practiced or planned \ac{FDI} and only 24.7\% chose Foreign Outsourcing, transferring tasks to external partners. Most often, a new subsidiary had been established (47.5\%). (\cite[p. 18]{StatistischesBundesamt.2008})



\subsection{Significant Differences between Germany and the USA}
\label{sec:DifferencesUSGER}
As shown in the previous sections, there are vast differences between Germany and the USA when it comes to offshoring. However, accurately quantifying those differences is no small effort. Government institutions for measuring trade activities exist in both countries, but there are no international standards regarding the indicators. Furthermore, both economies vary largely in size. The U.S. are the worlds largest economy with a \ac{GDP} of \$17.947 trillion in 2015, while Germany had a \ac{GDP} of \$3.356 trillion\footnote{Data source: World Bank, \url{http://data.worldbank.org/indicator/NY.GDP.MKTP.CD?locations=US-DE&start=1990}, visited on 14. August 2016}. Therefore one can not simply compare unadjusted offshoring volume. Additionally, currency conversion is to be considered. 

\subsubsection{Maturity of Offshoring}

\subsubsection{Offshoring Locations and Distances}
94\% of American Offshoring Destinations: ``Farshore'', Germany: 52 Near, 48 Farshore \cite[pp. 175f]{Hutzschenreuter.2007}
