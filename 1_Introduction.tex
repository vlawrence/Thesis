% !TEX root = Thesis.tex
\section{Introduction}
\label{sec:Intro}
In the past two decades, globalization has changed the structure of markets worldwide. Whole countries specialize and build their strengths. For example, India has developed into the global IT development and support powerhouse to the extent that many companies from other countries have relocated these tasks to either their own Indian subsidiary or an external Indian company.

Germany is traditionally a country with very strong exports. Therefore, the German economy has profited from globalization. But high labor cost in Germany is a problem for many companies. They tried to solve this in a similar way many American companies have successfully lowered their labor cost, through shifting tasks to a different country with lower labor cost. This is called offshoring.

One of the main drivers for offshoring is cost, but other considerations such as the development of new markets, strategic concerns or fiscal incentives play also a big role in inspiring companies to offshore. In figure \ref{fig:GERMotives}, the results of a survey conducted by German \textit{Statistisches Bundesamt} regarding the motives of German service companies for offshoring are illustrated. Labor and other cost are two of the top three arguments for offshoring, considerations of strategic or fiscal nature are less important. The least relevant motivators are less business regulation in the offshoring destination and following customers or competitors offshore.

\vspace{5mm}
\bildcite{GER_Motives}{Motives to offshoring of German service companies}{fig:GERMotives}{Data source: \cite[pp. 26f]{StatistischesBundesamt.2008}}{0.7}


By the German as well as the U.S. general public, offshoring plans are often dismissed as a selfish move of profit-hungry corporations. Employees fear for their job security and politicians criticize the job loss that offshoring entails. Contrary to these objections, research has shown for the U.S. that there is no evidence indicating that jobs are shifted abroad in a quantity that is discernible on nation-wide statistic trends (\cite[p. 7]{Jackson.2013}). Similarly, in 2005 only 7.2\footnote{Figures are given with a decimal point.} \% of total German job losses were connected to offshoring (\cite[p. 30]{Gorg.2011}).

The survey mentioned above even found a positive employment effect for qualified jobs in the service industry. For the explored time frame, 20\% more jobs have been created than those that have been offshored previously. For less qualified jobs, 75.9 \% of shifted jobs have been compensated through the creation of new jobs. (\cite[pp. 21f]{StatistischesBundesamt.2008})

This shows that the fear of the general public is not warranted, even though offshoring may certainly have a profound effect on individual persons. But on the larger scale, offshoring improves productivity of countries (\cite[pp. 90f]{Jahns.2007}).

Additionally to resistance to offshoring from German society, especially from unions and worker's councils, offshoring ventures of German companies have often faced  problems and project failures. In contrast, American companies have been more successful.

There is a plethora of existing research of offshoring, separate for German and American companies, but rarely are both directly compared (one example is \cite{Hutzschenreuter.2007}). The reasons for existing differences are even less explored.
\vspace{3mm}

The goal of this thesis is answering the question:

\begin{quote}
	\centering \textbf{What are the differences regarding IT offshoring between German and U.S. companies, and how can they be explained?}
\end{quote}
\vspace{3mm}

In order to achieve this goal, the thesis will include an analysis on available literature on the topic. Research will be conducted in the libraries of \textit{Hochschule für Technik, Wirtschaft und Kultur Leipzig} and \textit{Ludwig-Maximilians-Universität München}. Additionally, online resources like \textit{Google Scholar} or \textit{SpringerLink} will be utilized to find relevant literature. 

The findings of this analysis can be found in section \ref{sec:Theory}. In this section, relevant terms will be defined, before exploring factors for offshoring in general and specifically for the U.S. and Germany. The last subsection contains a direct comparison of both countries.

The results of theoretical research will be extended through the conduction of expert interviews with interview partners, that have extensive experience with offshoring in the IT industry from different points of view. This empirical research will complement the theoretical findings with real-world knowledge and experience. Further reasons for the different offshoring results of U.S and German companies will be uncovered, adding new, unique knowledge to the existing body of research.

In section \ref{sec:Empiry}, the interview process is described in order to enable further research achieve comparability to the work that is described in this thesis. The following four subsections describe the results of the expert interviews, each subdivided into a section about the expert's background, a section describing the results of the interview and a final section with conclusions that can be drawn from the interview. The final subsection summarizes the findings from all interviews.

The scope of the empirical part of the thesis is limited to the IT industry for reasons of clarity and availability of experts for interviews. In the theoretical section \ref{sec:Theory}, data is analyzed for all industries.

In this thesis, production is not considered a part of offshoring. As detailed in section \ref{sec:DefTerms}, offshoring is a shift of tasks and does not involve shipment of physical goods. Furthermore, offshoring with the exclusive goal of tax evasion is not a subject of this thesis.

