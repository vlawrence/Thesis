% !TEX root = Thesis.tex
\begin{appendix}
\section*{Appendix}
\addcontentsline{toc}{section}{Appendix}
\label{Appendix}

\tocless\section{Interview Structure}

\label{app:InterviewStructure}

{\bf Introduction [10 Minutes]}

Hello, thank you for participating in this expert interview! I’d like to preface with a short introduction to what my thesis is all about. However, before we start I need your consent to me recording this conversation. Do you agree with recording the interview?

 – Wait for answer –
 
Thank you.

First, let me introduce myself. My name is Veronika; I’m currently in the last leg of studying Information Systems and working on my Bachelors’ Thesis. This thesis is about comparing offshoring approaches in the US and Germany. 
The following questions are all about learning as much as possible from your experience, so please take the freedom to answer as detailed as you deem appropriate.

First of all, I’d like to learn something about you. Please introduce yourself and tell me about your international working experiences.

{\bf Offshoring   Experiences in the US / with the US [10 Minutes]}
\begin{itemize}
	\item In what way did you experience offshoring in U.S. companies? (Internal / Provider)
	\item In your experience, how do U.S. American companies approach offshoring?
	\item How is the working relationship between the US and the offshoring destination?
	\item If you think about offshoring in U.S. companies, is there any significant anecdote you’d like to share? Why is this a typical situation in this context?
\end{itemize}
	
{\bf Offshoring Experiences in Germany [10 Minutes]}

\begin{itemize}
	\item In what way did you experience offshoring in German companies? (Internal / Provider)
	\item In your experience, how do German companies approach offshoring?
	\item How is the working relationship between Germany and the offshoring destination?
	\item If you think about offshoring in German companies, is there any significant anecdote you’d like to share? Why is this a typical situation in this context?
\end{itemize}


{\bf Comparison [10 Minutes]}
\begin{itemize}
	\item In your opinion, what are the most significant differences between US American and German companies when it comes to offshoring?
	\item Further questions to clarify points as needed
\end{itemize}

{\bf Finalization [5 Minutes]}
Thank you again for taking the time to answer my questions today. It was a great help! Is there anything you would like to add, or any feedback you might have regarding this interview?

It was great to learn from your experience today. I’ll be in touch should there be any points that need further clarification, is that all right for you?

Thank you again, have a great day/evening/weekend!


\tocless\section{Interview Summaries}
The expert interviews are summarized based on the recorded .mp3-files. There may be gaps in the summaries, when there is no relevant discussion or breaks caused by external influences. All interview recordings have been added to the appendix on a CD and are considered the primary source.

\tocless\subsection{Michael Scheitza, 07/01/2016}
\label{int:Scheitza}

\begin{longtable}{l p{12.5cm}}
		\textbf{Time} & \textbf{Summary} \\ 
		01:00 -- 01:55& Introduction and consent to recording\\
		01:55 -- 02:49& Michael Scheitza has worked for eight years with different offshore approaches. He has experience with Russia, Poland, Romania, India, Malaysia, Mexico and Brazil. The longest projects he had with Russia, Romania and India.\\
		03:45 -- 03:54& He has worked for a few weeks in Malaysia and India. In Poland, he worked for half a year, but that was not for an offshoring experience.\\
		03:54 -- 04:24& He has no experience with offshoring from an U.S. American point of view, so this part of the interview is skipped.\\
		05:22 -- 08:05& At T-Systems, application management contracts work well with offshoring, provided there's no legal obligation to deliver locally. Most customers leave the choice of location of delivery to T-Systems. The delivery model is usually decided by needed skills, requested language and required service levels (pertaining to time zones).\\
		08:05 -- 09:35& Knowledge is not the only factor in deciding on a delivery model, but scalability is also very important. For a project, there need to be enough people with the required knowledge. When this can't be ensured, a different point of production must be chosen.\\
		09:47-- 10:45&Working relationship between T-Systems and the offshoring partner depends on the type of contract. There is an example given of an application management deal with Brazil, which contained many small applications. This meant that the team size was about 20 people, all of which were requested to speak enough German to directly interact with the customer.\\
		10:45 -- 11:43&In the transition phase of the project, the Brazilian team came to Germany in order to get the needed knowledge directly from the customer. In this time, one-on-one relationships between the Brazilian team, the customer and project management in Germany were established. This facilitated collaboration later on because people knew each other in person and not only via email and telephone.\\
		11:43 -- 12:45&In larger deals that involve a larger team, such deep collaboration is usually not established. Instead, the working relationship is managed via \glspl{sla} and \glspl{kpi}, where quality and quantity of deliverables are defined.\\
		12:46 -- 13:57&Neither approach is clearly superior to the other (personal collaboration vs. management via \gls{sla})\\
		13:57 -- 15:38&He had an experience once with an Indian Team, where money was spent on bringing people to Germany to improve collaboration and quality. Few months later, these people ended up leaving the project to further their careers, because having worked abroad is an achievement that enables people to earn more in India. So the money spent on improving collaboration was essentially burned.\\
		15:38 -- 17:09&In the first three months, it is good to build personal relationships with team members. In the long run, there are two options. One option is the really deep personal exchange outlined in the example of the Brazilian team , which has the downside of increasing volatility in the team and is not a standard approach. The other option is to draw motivation out of the contract and out of being successful in fulfilling the contract. \\
		17:12 -- 19:01&Personal relationships are very important for employee satisfaction, but there are two possible identifications for people working offshore for a project: one is the identification with the project itself and being motivated by the local team lead. The other possibility is getting into the personal relationship with the customer (can be both T-Systems and the end customer) and identifying as part of a team.\\
		19:01 -- 19:25&Such identification with a global delivery team is not possible in large teams (50+ persons), in his experiences.\\
		19:25 -- 20:15& If the onsite and the offshore team share the same tasks (``Verlängerte Werkbank''), the team size is usually less than 30 people. The project manager is then distributing tasks directly to offshore team members.\\
		20:15 -- 20:36 &If the team is large enough to be organized into different organizational layers, e.g. local project managers or team leads, these personal relationships get lost.\\
		21:05 -- 22:22&There is the clich\'{e} that in the US, there is a certain motivation culture that involves a lot of enthusiasm, whereas in Germany, there is a lot of focus on the organization and the end result. Both have a certain truth to them but do not cover reality. Similarly, in general people are happier when working in an integrated way in an offshore team. The prerequisite is that the tasks enable this working mode.\\
		25:00 -- 26:47&In smaller scale collaborations, it is important to know the people you are working with on a personal level, not only by a name and picture. Especially in Munich, he has hosted so many offshoring partners that he is now one of the best tourist guides. He shows them the sights in order to let his guests learn about our cultural background and to start a discussion. This is helpful in building personal relationships.\\
		27:55 -- 28:50&Thanking the interview partner and finalization\\
		
\end{longtable}

\tocless\subsection{A.S. Viswanathan, 07/07/2016}
\label{int:Viswanathan}

\begin{longtable}{l p{12.5cm}}
	\textbf{Time} & \textbf{Summary} \\ 
	00:34 -- 02:54 & Introduction and consent to recording\\
	02:54 -- 04:24& Viswanathan is electrical engineer with a specialization in industrial engineering. In 1978, he started his career with English Electric which was a part of the General Electric Group. He worked there for two years, then he changed employers and started with Siemens. He held several positions, from the shop floor to CIO of the IT subsidiary of Siemens in India. Later, he moved on to the board of Siemens Information Systems, a software company that took global mandate within the Siemens Group.\\
	04:24 -- 06:00 & His responsibilities with Siemens were primarily the Business Solutions, as well as pioneering offshoring SAP with his team. Furthermore, he was responsible for IT services. In 2007, Siemens merged all local IT companies (mentioned are India, Germany, Austria, Switzerland, and Greece) into a new company called IT Services and Solutions. Viswanathan was on the executive management of this company and headed Global Portfolio of Mobility which included Transportation and Logistics on water, air etc.\\
	06:00 -- 07:00 & After taking a break in 2011, he founded his own management consultation company in 2012 with primarily customers from Germany, China and India.\\
	07:00 -- 08:20&When they conceptualized offering offshoring services the first customers were from the U.S. and the UK. They were very quick in understanding the cost advantages of offshoring and seizing the opportunity, not only shifting single tasks, but the entire operations to India. Customers in the U.S. were fairly open to offshoring. Viswanathan had the feeling that they would just go ahead and implement offshoring, so there were not many problems.\\
	08:20 -- 09:34&Working relationship with the U.S., in his experience, have been positive. Contributing to that was English being a common language and management meetings and schedules were easily set up. The \glspl{sla} were more critical. A reason for the positive experience could be that U.S. companies had virtually no plan B with respect to offshoring. When they did offshore, they took the time to evaluate different vendors, but once the decision was made, they just went with it and the work was shipped to India. This means there was a necessity to make the relationship work.\\
	09:34 -- 10:14& After the initial cycles of new contracts, the focus shifted to delivery. There would be a next wave of requirements with better productivity standards. In this phase, the facts and figures dominated working relationships, especially \glspl{sla}.\\
	10:14 -- 11:40&The American approach to offshoring is characterized by legalistic and contractual considerations. They always have consultants as part of the team who would do a good background search of the companies that provide operations in India. Then they would select three to four companies, travel to India and go to presentations of the chosen companies and then enter contract negotiations. They spent a lot of time on the contracting, so on the commercial \glspl{sla} and not so much on the processes. They believed it would be done and depended on that.\\
	11:40 -- 13:46&When it came to bringing Indian onsite teams to the U.S., for example for transitions, visa issues caused an unexpected amount of problems. This went so far that it became an administrative aspect of the discussion of offshoring with new customers.\\
	13:46 -- 15:03& The offshorability of an American job is very high. The work is conceptualized in a way that the needed skills can easily be managed. It is not a holistic job, but a specific set of tasks where a specific set of skills is needed. This goes back to U.S. companies already having experience in shifting tasks, but within the U.S.. People are working in different places and different time zones within the U.S., so they were already used to working in such a way.\\
	15:03 -- 16:39 & Jobs are very transaction-based in the U.S.. The education system facilitates that the single employee does not need to have an end-to-end knowledge of the entire process or the bigger picture of why this process works in that way. This is connected to the higher fluctuation of employees in American companies. Simultaneously, it is a huge advantage when it comes to offshoring. It would only need some kind of minimal training to get a new person for the job up and running.\\
	16:40 -- 18:24 & In German companies, the job design is more intrinsic and process-oriented. For example, a buyer in Germany compared to a buyer in the U.S. has a more specific background, maybe an apprenticeship in the field or some kind of special training. So when it comes to customizing an SAP system, a German system would differ greatly from an American system, because the role of an individual is more process- and system-oriented and holistic in Germany.\\
	18:24 -- 19:30& This role system presents a problem when shifting tasks offshore, because it can't be transferred like to like. That means, one person in Germany cannot simply be replaced with another person in India, because that person lacks the specific background and experience with the German company. Also, knowledge requirements are very intricate. In the buyer example, the employee needs knowledge in costing, the market, product design and so on. They wouldn't consult a technical specialist for those details. This system-oriented thinking is an advantage of the German society, but also an obstacle for offshoring.\\
	19:30 -- 19:55& Employees in Germany have proper education, vocational training and guidance and are very competent when they start the job. This competence is mirrored in the SAP systems and when the SAP system is then moved to India, there is a problem because there are no employees with the same skill set there.\\
	19:55 -- 21:00& Offshoring is a very alien concept to Germany, especially for \textit{Mittelstand} companies\footnote{\Acrlong{sme}}. Those grow very organically from family businesses, so they have in-built control of all processes. When it comes to outsourcing within Germany, the service provider does not have the full control of the process, they only provide some parts. But when it comes to offshoring, control must be given up by the customer, which is a very foreign concept for German companies.\\
	21:00 -- 22:28&Even when it comes to replacing domestic outsourcing with foreign outsourcing, the customer would complain about the offshore service provider, because the local service provider had a very good knowledge of their company. They were entirely dependent on the local service provider, almost like they were an extension of their own company to the extent that if there would be a larger incident, the provider's employees would come running even outside of their normal service hours. Of course, this can't be replicated with an offshore service provider.\\
	22:56 -- 24:13& Germans display a high degree of detailing, so the standards for the service providers tend to be very high. This has been one of the biggest barriers, and it is important to change the mindset regarding this. Also, Germans tend to prefer \gls{fdi} over foreign offshoring.\\
	24:13 -- 25:15& He did recently consult a German company with setting up \gls{fdi} and remarked to them that this practice is very typical for German companies. Frequently, the objective is recreating the own organization in India (``Mini-Germany in India'')\\
	25:15 -- 26:09&From a business solutions perspective, Germans would love to have the cost arbitrage, but they need to have the cost arbitrage the way the jobs are designed.\\
	26:09 -- 26:31&Another important point is language. Germany has become more international, but many companies still prefer to have their systems built and maintained completely in German. This is not essentially wrong, because many Indians are learning German as well, but it becomes a handicap. English is more easily learned.\\
	26:31 -- 28:28& Germans are more used to exporting then to importing, so they are not used to other countries selling to them or being better than them in the provision of services. Also, there's the issue of managing the transition with the loss of jobs or the decrease of job security when implementing offshoring. \textit{Betriebsrat}\footnote{Work council} and unions make this process slow and difficult, whereas it is much easier in the U.S.. \\
	28:28 -- 29:14&This needs to be accepted and accounted for in the planning of offshoring, both on the side of Germany and India. An American company would be much quicker in making the transition to offshoring, which is a disadvantage for German companies.\\
	29:14 -- 29:39&The reason for German companies to choose \gls{fdi} over foreign outsourcing is the inability to change the structure of jobs in a way that makes them easily offshorable (``slice and dice'').\\
	30:30 -- 33:09& A German company which wants to make the most of offshoring needs to think of entire functions in a process or in the company they would offshore, rather than offshoring just some minor roles. The second thing is, IT services can easily be offshored, meaning hardware, infrastructure support and similar tasks. But when it comes to process design, they can divide the job into a different level, so they can separate the decision-making part and the transaction part. Then, the transactions could be done elsewhere without many problems. But as long as processes are looked at in an integral way, there is a problem. So, the very deep level of job slice and dice that is common in America is not needed, but that one level of division could help German companies a lot.\\
	33:09 -- 33:33& This is one of the biggest issues, because at the moment one person makes the decisions but also posts data sets into SAP modules.\\
	33:33 -- 35:00 & With one customer, shortly before completing the implementation of offshoring they had to send back the work because unions had objected to the project. This was a very difficult scenario for both the service provider and the customer, as the setup in India was already completed.\\
	35:00 -- 36:48 & Application management was identified as one function that is easily offshorable. Change management with respect to other functions was more difficult, but application management could be dealt with by replacing one German resource with three Indian resources. This resulted in a much smaller cost arbitrage in an one-on-one comparison, but as multitasking was done in the Indian site, it yielded economic advantage. The distribution of work was here managed by Indian managers who had enough knowledge of the entire process. This is a lot more management effort than offshoring with American companies would entail.\\
	36:48 -- 38:49& Thanking the interview partner and finalization\\ 
\end{longtable}
\newpage
\tocless\subsection{Ingo Kümmritz, 08/04/2016}
\label{int:Ingo}
\begin{longtable}{l p{12.5cm}}
	\textbf{Time} & \textbf{Summary} \\ 
	01:23 -- 02:16 & Introduction and consent to recording\\
	02:16 -- 03:35 & Ingo studied in the U.S. (High School and College). This has helped him to broaden his horizon when it comes to international delivery. In 2003, when he was working at IBM, the country manager approached him, asking if he could drive the topic of global delivery. After 10 years of working at IBM, he switched jobs and worked at Siemens. Via a short contract with an Indian company he ended up at NTT \nolinebreak Data in Germany.\\
	03:35 -- 04:50&He has been working in the area of global delivery for 13 years, mostly with India (90\%). In this time, he was responsible for big projects and \gls{ams} deals. The critical point, regardless of the subject of the work, is communication and cooperation. It is necessary to learn how the counterpart on the service provider side is thinking and reacting to communication. So interaction between the partners is key, rather than processes, methods and tools.\\
	04:50 -- 06:08&Still, processes, methods and tools are the basis for any successful work, be it in Germany or in an international context. Twelve years back, he used to be in the role of a principal, which is a topic expert (as opposed to a people leader). Within Siemens, he moved on to a customer-facing role. At present, he holds a subject matter expert role again with NTT Data. So he has experience in delivery, sales, and customer-facing positions, which helps him understanding the partners involved.\\
	06:08 -- 07:21&On the question whether he was involved in offshoring from a customer point of view, he answers ``yes and no''. He explains that when talking about global delivery, there is no dedicated location for delivery, but rather a delivering company. This company then needs to find the right delivery model, concerning resources and locations. This is not necessarily India, but this country is the powerhouse in the industry. So he has been both on integrated delivery teams and on customer teams that travel to India to set up offshoring with a company there.\\
	07:20 -- 08:10 & It is clarified that his experience pertains to \gls{fdi}. Whenever he has worked with an Indian team, they were his colleagues at IBM or Siemens.\\
	08:22 -- 09:00&Although he did not have first-hand experience with offshoring in the U.S., he spent time there and had colleagues there, considering IBM is an American-based company. However, the German subsidiary would probably considered offshoring by the American headquarter, so in this way, he has experience in offshoring for a U.S. company.\\
	09:00 -- 10:06& India is a preferred partner when it comes to offshoring because the time zones are very convenient. Central European business hours can be covered from India without needing late shifts from the employees there.\\
	10:06 -- 10:46&Even though he has no direct experience with offshoring from a U.S. perspective, he feels comfortable to share what he has learned from his American colleagues. He thinks there is a significant difference between German and American approaches to offshoring.\\
	10:46 -- 12:58& There are two main drivers behind offshoring in the U.S., to Ingo's knowledge. One of those drivers is cost. An administrator in the U.S. works for cost level X, while an Indian colleague would cost a fraction of X. For American companies, this cost arbitrage seems very tempting. The willingness to take risks in order to take advantage of the cost arbitrage is a lot higher than in German companies. Since there is no language barrier between the U.S. and India, and the U.S. being a civilization that Indian companies like to be working for, there are not many barriers to American companies that want to offshore.\\
	12:58 -- 14:25& The second point is, when looking at companies not involved in the IT industry, there is a high volatility in their business behaviour. That means, often there need to be large teams set up on short notice and the companies would rather not onboard so many people as the needed skills are often uncommon, they are just interested in the results. This increases the willingness of companies to shift tasks offshore.\\
	14:25 -- 15:13&Ingo gives the example of DHL, before the company was bought by Deutsche Post. DHL employed roughly 4000 Indians from Infosys, an Indian-based company, in this time. Then, Deutsche Post bought the company, the service providers were consolidated and did not include Infosys any more. This is just a different approach when dealing with new ventures, business plans in the U.S. almost always include out-tasking certain areas.\\
	15:13 -- 16:49&Another point is, that American companies have a much higher tolerance for software code that is not 100\% perfect, but performs fast, than engineering-based civilizations such as Germany, Switzerland or Nordic countries.\\
	16:49 -- 17:22& This is a sign of the dedication to getting products to the market quickly. American companies are not as worried about failures as German companies. If a new product does fail at an American company, they learn from it and start over, whereas German companies try to avoid failures.\\
	17:22 -- 18:30& There are two main possibilities of working relationships between American companies and their Indian service providers. One is handing the tasks over and managing the service provider as pure subcontractor. This is a more numbers-based working relationship and not much of a partnership or an involvement on a technical level.\\
	18:30 -- 19:40& For more sophisticated set-ups, there is the approach of an integrated team. Here, you have on-site staff, people from India on-site (they are called landed), and people offshore, so this is a three-tier delivery model. There are defined roles and rotations. In this delivery model, there is more of a partnership between the customer and the service provider.\\
	19:40 -- 20:30& Both of these approaches work very well in the appropriate circumstances. To some American companies, it seems natural to work in distributed teams with English-speaking Indian colleagues, while there are problems working with South America.\\
	20:30 -- 21:09& From the customer's perspective, there is the third possibility of contracting a global delivery company and collaborating only on a strategic level. The service provider then decides on delivery model and locations.\\
	21:09 -- 23:25& American companies are better at utilizing dedicated offshore teams. When offered a dedicated offshore center, American companies will get very creative in finding extra work for the offshore team to keep them occupied, maybe some backlog tasks that keep being postponed because there are more urgent issues at hand. Germans are very hesitant with this and prefer to ``pay per use'' of offshoring resources. Ingo thinks that this is something that needs to change in Germany.\\
	23:25 -- 24:16& This hesitation of German companies to use delivery centers to full capacity might stem from a lack of trust on the German side. Also, Germany has a risk-avoiding culture which may be aggravating this.\\
	24:16 -- 24:59&Ingo clarifies that he does not want to imply German companies are not taking any risks, but there is a tendency to have several rounds of risk-assessment and the need to collect as much information as possible in order to correctly evaluate the risk, which slows down the process. In America, decisions are made much quicker and in a ``hand-shake business'' manner. So there is an ability of instant execution.\\
	24:59 -- 25:40& The issue with the delivery centers is that American companies approach it with a level of trust, but German companies do not want to pay for 100 people when they only have work for 20. American companies are willing to pay for the reserved manpower in case something urgent comes up; otherwise, a backlog of work that may be not in the original scope gets moved offshore.\\
	25:40 -- 27:28& When Ingo was working for IBM, he wanted to get some Indian SAP experts for a new project. When he approached the head of the SAP resources in India, he told him there was no one available, as of the 250 persons there, 50 were in existing projects, and 200 were reserved by the American part of IBM for projects to come. That was why Ingo could not get any Indian resources for his project, and he thought it was incredible to put such manpower on the bench. Nobody from the German part of IBM was willing to work likewise, ramping up an Indian team months in advance in order to be able to engage once the project started.\\
	27:28 -- 28:47&There has been this myth in Germany that Indians are great and able to deliver any task just based on the requirement documentation. Of course, they will deliver, and considering that in this time 300 000 new IT engineers approached the Indian job market, it is just a matter of statistic that there also were quite a few very good IT engineers, experts and consultants.\\
	28:47 -- 29:45&However, German companies never learned to deal, on a partnership level, with the Indians. They approached them like they would have done in a German factory without keeping cultural differences in mind. They would just hand over the documentation and check the results a few weeks or months later. That does not work with India.\\
	29:45 -- 30:20 & At a company he experienced this, after a badly failed software engineering project, the German team came to the conclusion that they should talk to the Indian service provider once in a while during the project, which mostly also resulted in failures. When the Indian colleagues were asked if they understood the documentation, they assured that they did, but the project still failed.\\
	30:20 -- 31:25& The third wave of learning included looking into the cultural background of the Indian colleagues and trying to understand them better. When asked to perform a task, they would confirm they could do it because they did not to fail their customer, regardless if they could actually do it or not. Germans would protest if they were asked to perform a task in an impossibly small time frame. This relates to the German engineering background, including buffers in the planning in order to deal with anything unexpected.\\
	31:25 -- 31:54&Once the German team learned how to hand over tasks to India, check for results and employ trust, they could build an integrated team with the Indian colleagues and then, the projects succeeded. Such an integrated team continues to succeed for a long time, because this mutual level of trust and understanding was a great motivation for all participants.\\
	31:54 -- 33:18&The trouble with German companies in India is that many do not have a brand there and are not generally known. Indians have adopted the American way of building their careers and CVs based on the reputation of their employers. So German companies have a hard time attracting enough skilled resources when setting up \gls{fdi}. Instead, Indians loved to join Dell, IBM, HP, Accenture or one of the top-tier Indian providers.\\
	33:18 -- 33:55 & German companies have needed a long time and lots of experience to come up with a way to deal with offshoring, to implement checks and balances, to have a communication plan, and how to deal with the Indian colleagues. Once these are in place, offshoring can work very well for a long time in German companies.\\
	33:55 -- 34:41&From the Indian side, service providers have learned about Germany as well, founded small centers in Germany, and hired local people who could make a bridge between India and Germany. In order to become successful, they want to understand their customer's culture better.\\
	34:41 -- 36:20&Of course, there is the language issue as well when talking about offshoring in Germany. Volkswagen, for example, has a lot of IT services, and everything has to be in German. So they have large, multi-million euro projects, completely in German, and they also insist on large teams, sitting right next to their premises in Wolfsburg. This is virtually impossible because Wolfsburg is not a very desirable city. There are only so many German-speaking IT engineers, but with Volkswagen requirements, 100 people in a team of 150 need to speak German, and that is a huge problem.\\
	36:20 -- 37:55 & BMW, on the other hand, is more international and even requires Chinese in some positions, because the Chinese market has become very important for them.\\
	37:55 -- 39:38& One obstacle to German companies becoming more international is many German people not being confident in their English skills. This goes back to trying to not make any mistake when using a foreign language. Also, they fear being ripped off or losing in negotiations.\\
	39:38 -- 41:00&Ingo's impression is that the integrated team approach is much more common in Germany. Bank institutions were the first to try and benefit from the cost arbitrage, but they soon found out that offshoring comes with additional management overhead, so their own experts did not get as much relief as anticipated. When they switched to integrated teams, they could reduce this communication gap, even though it cost more.\\
	41:00 -- 43:21&From that, German companies have learned that just shipping tasks to India does not work and prefer integrated teams for that reason. Ingo thinks, this approach is also more successful. This may go back to the desire to understand how things work which is a German specialty. In contrast, U.S. companies simply do not care how results are achieved, so they are more \gls{kpi}-driven. This desire to understand makes things slower.\\
	43:21 -- 45:47&If we look at typical German behaviors at meetings, there is a round-table mentality: everybody gets to say something. In the U.S., the decision-making process is well-prepared and results in a quick decision, which may be not as over-engineered as in Germany. However, the quality of decisions in Germany is often very good, as many people have contributed their knowledge. In the U.S., this may get lost.\\
	45:47 -- 48:50& It is astounding how many companies in Germany go through the same learning process with regard to offshoring, first shipping the tasks without much communication, failing and only then learning from their mistakes.\\
	48:50 -- 49:55&Ingo's Indian counterpart had the tendency to oversell their services, whereas Germans tend to be overly critical with themselves.\\
	49:55 -- 50:25& Ingo's key statement always is: ``If you are capable to combine German engineering and Indian enthusiasm, then you have a successful team.''\\
	50:25 -- 53:05& Thanking the interview partner and finalization\\
\end{longtable}

\newpage
\tocless\subsection{Subir Purkayastha}
\label{int:Subir}

\begin{longtable}{l p{12.5cm}}
	\textbf{Time} & \textbf{Summary} \\ 
	00:25 -- 03:06&Introduction and consent to recording\\
	03:06 -- 03:59&Subir did his undergraduate degree in engineering from India, then he went to the U.S. for graduate studies, initially in engineering. Afterwards he worked for a few years, before going back to graduate studies. He completed a PhD-degree in a Computer Science related field at the University of Michigan and his thesis was about data management systems. For 15 years, he worked at AT\&T Bell Labs where he developed network software products. Subir then joined Siemens, where he headed the telecommunications business unit in the IT division of Siemens India.\\
	03:59 -- 04:30&During the 16 years Subir worked there, Siemens build up the software division in India from 150 to 7000 people. In his business unit, he worked extensively with German and U.S. clients, which were both Siemens subsidiaries and external companies.\\
	04:30 -- 04:53&In 2001, he moved back to the U.S. and worked another ten years at Siemens, handling the sales marketing and delivery for the IT division of the company. In this time, he worked whit U.S. and South American clients.\\
	04:53 -- 05:15&After retiring from Siemens, Subir has started working at a local collage and managing the software group, which is with about five people really small. But at Siemens, he was responsible for about 500 people and a revenue of 100 million dollars a year.\\
	05:15 -- 05:55&German clients mostly have been Siemens divistions, whereas U.S. clients have been both Siemens division and external companies. Therefore, the comparison will be not completely like to like but he will try to generalize as much as possible.\\
	06:30 -- 07:09& In the U.S., the software development and delivery part of IT industry, the approach of companies over the last 20 years has been reducing cost.\\
	07:49 -- 08:30&Most of the business volume by U.S. companies is from financial services and retail industries. A different group of companies that have a strong focus on quality are the software development companies. They look not only at cost reduction, but software quality as well because for software products, quality is really important.\\
	08:30 -- 09:16&In software industry, 70\% of the work is routine support tasks and software development tasks. For the development of software products, quality is paramount. Subir has worked with a few software companies who chose the IT division of Siemens over the major name brands in India and the U.S. for their development, because they had a lot of experience in developing software products for the Indian and German divisions of Siemens, for example in car electronics or telecommunication products.\\
	09:16 -- 09:55&Compared to the U.S., Germany had a strong emphasis on quality. In U.S. companies, the planning cycles are a lot shorter than in German companies. That means, they try to make decisions quickly and go ahead and implement them. They have a lot less bureaucracy and their time frame from idea via decision to implementation is a lot shorter.\\
	09:55 -- 11:20&When offshoring a major operation, it could be support or development, a long planning time (at least 6 months) is needed. But in the U.S., companies often give only two to three months for transition. This creates all kinds of problems with change and transition management. Especially for the client's employees it is important to carefully manage the transition to offshoring, breaking the news in a gentle way, showing them their next steps and if possible and necessary help them find new jobs. U.S. companies are not good at this process.\\
	11:20 -- 12:10&As a consequence, existing employees are in a state of shock. This prevents them from collaborating with the service provider, who is under time pressure as well. This makes a smooth transition next to impossible. There is not enough time for the service provider to learn the processes or for a certain time of parallel operation. If the transition does not go well, there are a lot of problems in the first year of operation.\\
	12:10 -- 12:43&Subir thinks U.S. companies do not fully understand the complexity that comes with offshoring, particularly if there are a lot of people involved.\\
	12:43 -- 13:22& So the quick decision-making and implementation time of American companies are, in Subir's view, a big disadvantage when it comes to offshoring.\\
	13:22 -- 14:20&Of course, this is a generalization, and not meant to say that there are no companies with more experience or that are more progressive. Usually, offshoring decisions are made by high-level business managers. Technical managers are often against offshoring, are overruled in the decision making and their views are completely disregarded afterwards.\\
	14:20 -- 16:20& For example, a vice president of a division in the U.S. who has budget goals to meet sees that employing people in the U.S. is expensive. First, he may try moving the employees within the U.S. to a lower cost area, but when that does not work, the work is offshored to India or the Philippines, for example. Because business leaders in the U.S. are held responsible for budgets quarterly, the vice president is under a lot of time pressure to implement offshoring as quickly as possible, else he could lose his job. This corporate environment in the U.S. is not conductive to sustainable delivery.\\
	16:20 -- 18:10&The critical success factors for offshoring are a sufficient transition time frame (as previously mentioned) and the stability of the team who will deliver the service.\\
	18:10 -- 19:30&At Siemens, there was a certain delivery culture or philosophy: The stability of delivery teams was very important for them. The knowledge in the team was retained by the top 25-30 \% of the team. For example, a delivery team for a software development project consists of 20 people, so the top 25 \% are five to six people. These senior people collect all the knowledge about the product, the project and the customer while working in the project.\\
	19:30 -- 20:20& The other people in the project are software developers, testers and so on. Those can be replaced if need be, but the senior people are key to project success. So the critical success factor is keeping this part of the team stable, because the knowledge they have accumulated can not be replicated by new hires.\\
	20:20 -- 21:50 & At Siemens, these important employees were kept with the company by giving them good career options. This is not limited to money, but more in the sense of additional responsibility and recognition as a delivery leader as opposed to a software developer. Over time, they would learn more and more technologies and domain knowledge about customers and their industries. These delivery leaders are an invaluable asset when it comes to offshoring.\\
	21:50 -- 23:25&New employees start in a certain domain. After a few years, they can move to a different domain and broaden their knowledge. This has the advantage that new customers get the impression that their offshore software engineers understand their business. This enables the software engineers to ask the right questions and make informed decisions when working for this customer. This knowledge can only be built up in a company if there is a special focus on retaining senior people in the company.\\
	23:25 -- 24:30&Besides Siemens, several Indian service providers have recognized the importance of knowledge in the companies and introduced similar policies. The key, senior employees have to be paid well, their careers must be planned and they have to be taken care of in order to keep them in the company. This costs time, effort and money, so the cost structure of such companies is higher than in companies where one senior employee supervises 20 juniors.\\
	24:30 -- 25:22&If the focus of the customer is only on cost, companies with only one senior employees in a team of 20 people have an advantage. Those companies usually have a high fluctuation of junior people because they are trying to further their careers and increase their salary quickly.\\
	25:22 -- 25:59&During presentations, Subir and his team tried to emphasize the fact that they have a lot of domain knowledge in their teams, but most of the American clients only cared about the bottom line price.\\
	25:59 -- 26:35&The third success factor for offshoring is the time the customer is willing to spend on explaining the business and the processes. In order to facilitate this, it was common to invite some employees of the client to India and spend three to six months to explain the business and work as part of the team. This helps to develop a good relationship.\\
	26:35 -- 27:12& If this is not possible because of cost issues, the client should at least visit their delivery team once or twice a year.\\
	27:12 -- 28:38 &This costs time and money, but helps to motivate the delivery team. Also, the service provider would send their project manager and delivery experts to the customer for two tasks: requirement definition and system testing.\\
	28:38 -- 29:40&This level of engagement and spending time with each other is a very critical success factor in offshoring. German Siemens divisions were more inclined to put in the effort than U.S. Siemens divisions or external clients. All this travel costs money which increases the bottom line price, but it is a good investment in project success, because requirements are understood better, the working relationship is better and there is more motivation for the delivery team. Once the product is moved in production, the quality is much better.\\
	29:40 -- 30:00&That is what is needed for a successful offshoring project. If the focus of the client is only cost, this becomes much harder to sell. If the client is more progressive, they understand the importance of quality and long-term relationship.\\
	30:30 -- 32:40&A fourth success factor is having senior people on the client's organization and their view on the offshore delivery team. So Subir used to organize town hall meetings with the client's business leaders and the delivery team to make the offshore team see who they are working for and feel as part of the client's team. This is of course much easier when working for other Siemens divisions.\\
	32:40 -- 33:15&Even for outside clients, this is really important. It must not be just a lip service, because offshoring is not just an instrument in cutting cost, but an extension of the client's own team. Everybody involved needs to be on the same page, that in a global economy, global delivery and service is needed.\\
	33:15 -- 34:55&In Subir's view, U.S. companies have poor transition planning and think that the stability of the delivery team is the problem of the service provider. With investing client's time, Subir has had mixed results. Companies with a focus on quality understood the need to spend time with their supplier. Only Siemens divisions believed that the offshoring team is an extension of their own organization. External customers were using offshoring just as a cost-cutting tool.\\
	
	
\end{longtable}	
\end{appendix}	