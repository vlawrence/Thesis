\begin{appendix}
\section*{Appendix}
\addcontentsline{toc}{section}{Appendix}
\label{Appendix}

\tocless\section{Interview Structure}
\label{app:InterviewStructure}
{\bf Introduction [10 Minutes]}

Hello, thank you for participating in this expert interview! I’d like to preface with a short introduction to what my thesis is all about. However, before we start I need your consent to me recording this conversation. Do you agree with recording the interview?

 – Wait for answer –
 
Thank you.

First, let me introduce myself. My name is Veronika; I’m currently in the last leg of studying Information Systems and working on my Bachelors’ Thesis. This thesis is about comparing offshoring approaches in the US and Germany. 
The following questions are all about learning as much as possible from your experience, so please take the freedom to answer as detailed as you deem appropriate.

First of all, I’d like to learn something about you. Please introduce yourself and tell me about your international working experiences.

{\bf Offshoring   Experiences in the US / with the US [10 Minutes]}
\begin{itemize}
	\item In what way did you experience offshoring in U.S. companies? (Internal / Provider)
	\item In your experience, how do U.S. American companies approach offshoring?
	\item How is the working relationship between the US and the offshoring destination?
	\item If you think about offshoring in U.S. companies, is there any significant anecdote you’d like to share? Why is this a typical situation in this context?
\end{itemize}
	
{\bf Offshoring Experiences in Germany [10 Minutes]}

\begin{itemize}
	\item In what way did you experience offshoring in German companies? (Internal / Provider)
	\item In your experience, how do German companies approach offshoring?
	\item How is the working relationship between Germany and the offshoring destination?
	\item If you think about offshoring in German companies, is there any significant anecdote you’d like to share? Why is this a typical situation in this context?
\end{itemize}


{\bf Comparison [10 Minutes]}
\begin{itemize}
	\item In your opinion, what are the most significant differences between US American and German companies when it comes to offshoring?
	\item Further questions to clarify points as needed
\end{itemize}

{\bf Finalization [5 Minutes]}
Thank you again for taking the time to answer my questions today. It was a great help! Is there anything you would like to add, or any feedback you might have regarding this interview?

It was great to learn from your experience today. I’ll be in touch should there be any points that need further clarification, is that all right for you?

Thank you again, have a great day/evening/weekend!


\tocless\section{Interview Summaries}
The expert interviews are summarized based on the recorded .mp3-files. There may be gaps in the summaries, when there is no relevant discussion or breaks caused by external influences. All interview recordings have been added to the appendix on a CD and are considered the primary source.

\tocless\subsection{Michael Scheitza}

\begin{longtable}{l p{12.5cm}}
		\textbf{Time} & \textbf{Summary} \\ 
		01:00 -- 01:55& Introduction and consent to recording\\
		01:55 -- 02:49& Michael Scheitza has worked for eight years with different offshore approaches. He has experience with Russia, Poland, Romania, India, Malaysia, Mexico and Brazil. The longest projects he had with Russia, Romania and India.\\
		03:45 -- 03:54& He has worked for a few weeks in Malaysia and India. In Poland, he worked for half a year, but that was not for an offshoring experience.\\
		03:54 -- 04:24& He has no experience with offshoring from an U.S. American point of view, so this part of the interview is skipped.\\
		05:22 -- 08:05& At T-Systems, application management contracts work well with offshoring, provided there's no legal obligation to deliver locally. Most customers leave the choice of location of delivery to T-Systems. The delivery model is usually decided by needed skills, requested language and required service levels (pertaining to time zones).\\
		08:05 -- 09:35& Knowledge is not the only factor in deciding on a delivery model, but scalability is also very important. For a project, there need to be enough people with the required knowledge. When this can't be ensured, a different point of production must be chosen.\\
		09:47-- 10:45&Working relationship between T-Systems and the offshoring partner depends on the type of contract. There is an example given of an application management deal with Brazil, which contained many small applications. This meant that the team size was about 20 people, all of which were requested to speak enough German to directly interact with the customer.\\
		10:45 -- 11:43&In the transition phase of the project, the Brazilian team came to Germany in order to get the needed knowledge directly from the customer. In this time, one-on-one relationships between the Brazilian team, the customer and project management in Germany were established. This facilitated collaboration later on because people knew each other in person and not only via email and telephone.\\
		11:43 -- 12:45&In larger deals that involve a larger team, such deep collaboration is usually not established. Instead, the working relationship is managed via \acp{SLA} and \acp{KPI}, where quality and quantity of deliverables are defined.\\
		12:46 -- 13:57&Neither approach is clearly superior to the other (personal collaboration vs. management via \ac{SLA})\\
		13:57 -- 15:38&He had an experience once with an Indian Team, where money was spent on bringing people to Germany to improve collaboration and quality. Few months later, these people ended up leaving the project to further their careers, because having worked abroad is an achievement that enables people to earn more in India. So the money spent on improving collaboration was essentially burned.\\
		15:38 -- 17:09&In the first three months, it is good to build personal relationships with team members. In the long run, there are two options. One option is the really deep personal exchange outlined in the example of the Brazilian team , which has the downside of increasing volatility in the team and is not a standard approach. The other option is to draw motivation out of the contract and out of being successful in fulfilling the contract. \\
		17:12 -- 19:01&Personal relationships are very important for employee satisfaction, but there are two possible identifications for people working offshore for a project: one is the identification with the project itself and being motivated by the local team lead. The other possibility is getting into the personal relationship with the customer (can be both T-Systems and the end customer) and identifying as part of a team.\\
		19:01 -- 19:25&Such identification with a global delivery team is not possible in large teams (50+ persons), in his experiences.\\
		19:25 -- 20:15& If the onsite and the offshore team share the same tasks (``Verlängerte Werkbank''), the team size is usually less than 30 people. The project manager is then distributing tasks directly to offshore team members.\\
		20:15 -- 20:36 &If the team is large enough to be organized into different organizational layers, e.g. local project managers or team leads, these personal relationships get lost.\\
		21:05 -- 22:22&There is the clich\'{e} that in the US, there is a certain motivation culture that involves a lot of enthusiasm, whereas in Germany, there is a lot of focus on the organization and the end result. Both have a certain truth to them but do not cover reality. Similarly, in general people are happier when working in an integrated way in an offshore team. The prerequisite is that the tasks enable this working mode.\\
		25:00 -- 26:47&In smaller scale collaborations, it is important to know the people you are working with on a personal level, not only by a name and picture. Especially in Munich, he has hosted so many offshoring partners that he is now one of the best tourist guides. He shows them the sights in order to let his guests learn about our cultural background and to start a discussion. This is helpful in building personal relationships.\\
		27:55 -- 28:50&Thanking the interview partner and finalization\\
		
\end{longtable}

\tocless\subsection{Viswanathan ??}


\begin{longtable}{l p{12.5cm}}
	\textbf{Time} & \textbf{Summary} \\ 
	
\end{longtable}

\tocless\subsection{Ingo Kümmritz}

\begin{longtable}{l p{12.5cm}}
	\textbf{Time} & \textbf{Summary} \\ 
	
\end{longtable}

\tocless\subsection{Subir Pu???}

\begin{longtable}{l p{12.5cm}}
	\textbf{Time} & \textbf{Summary} \\ 
	
\end{longtable}	

\end{appendix}	