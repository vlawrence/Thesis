\begin{appendix}
\section*{Appendix}
\addcontentsline{toc}{section}{Appendix}
\label{Appendix}

\tocless\section{Interview Structure}

\label{app:InterviewStructure}

{\bf Introduction [10 Minutes]}

Hello, thank you for participating in this expert interview! I’d like to preface with a short introduction to what my thesis is all about. However, before we start I need your consent to me recording this conversation. Do you agree with recording the interview?

 – Wait for answer –
 
Thank you.

First, let me introduce myself. My name is Veronika; I’m currently in the last leg of studying Information Systems and working on my Bachelors’ Thesis. This thesis is about comparing offshoring approaches in the US and Germany. 
The following questions are all about learning as much as possible from your experience, so please take the freedom to answer as detailed as you deem appropriate.

First of all, I’d like to learn something about you. Please introduce yourself and tell me about your international working experiences.

{\bf Offshoring   Experiences in the US / with the US [10 Minutes]}
\begin{itemize}
	\item In what way did you experience offshoring in U.S. companies? (Internal / Provider)
	\item In your experience, how do U.S. American companies approach offshoring?
	\item How is the working relationship between the US and the offshoring destination?
	\item If you think about offshoring in U.S. companies, is there any significant anecdote you’d like to share? Why is this a typical situation in this context?
\end{itemize}
	
{\bf Offshoring Experiences in Germany [10 Minutes]}

\begin{itemize}
	\item In what way did you experience offshoring in German companies? (Internal / Provider)
	\item In your experience, how do German companies approach offshoring?
	\item How is the working relationship between Germany and the offshoring destination?
	\item If you think about offshoring in German companies, is there any significant anecdote you’d like to share? Why is this a typical situation in this context?
\end{itemize}


{\bf Comparison [10 Minutes]}
\begin{itemize}
	\item In your opinion, what are the most significant differences between US American and German companies when it comes to offshoring?
	\item Further questions to clarify points as needed
\end{itemize}

{\bf Finalization [5 Minutes]}
Thank you again for taking the time to answer my questions today. It was a great help! Is there anything you would like to add, or any feedback you might have regarding this interview?

It was great to learn from your experience today. I’ll be in touch should there be any points that need further clarification, is that all right for you?

Thank you again, have a great day/evening/weekend!


\tocless\section{Interview Summaries}
The expert interviews are summarized based on the recorded .mp3-files. There may be gaps in the summaries, when there is no relevant discussion or breaks caused by external influences. All interview recordings have been added to the appendix on a CD and are considered the primary source.

\tocless\subsection{Michael Scheitza, 07/01/2016}
\label{int:Scheitza}

\begin{longtable}{l p{12.5cm}}
		\textbf{Time} & \textbf{Summary} \\ 
		01:00 -- 01:55& Introduction and consent to recording\\
		01:55 -- 02:49& Michael Scheitza has worked for eight years with different offshore approaches. He has experience with Russia, Poland, Romania, India, Malaysia, Mexico and Brazil. The longest projects he had with Russia, Romania and India.\\
		03:45 -- 03:54& He has worked for a few weeks in Malaysia and India. In Poland, he worked for half a year, but that was not for an offshoring experience.\\
		03:54 -- 04:24& He has no experience with offshoring from an U.S. American point of view, so this part of the interview is skipped.\\
		05:22 -- 08:05& At T-Systems, application management contracts work well with offshoring, provided there's no legal obligation to deliver locally. Most customers leave the choice of location of delivery to T-Systems. The delivery model is usually decided by needed skills, requested language and required service levels (pertaining to time zones).\\
		08:05 -- 09:35& Knowledge is not the only factor in deciding on a delivery model, but scalability is also very important. For a project, there need to be enough people with the required knowledge. When this can't be ensured, a different point of production must be chosen.\\
		09:47-- 10:45&Working relationship between T-Systems and the offshoring partner depends on the type of contract. There is an example given of an application management deal with Brazil, which contained many small applications. This meant that the team size was about 20 people, all of which were requested to speak enough German to directly interact with the customer.\\
		10:45 -- 11:43&In the transition phase of the project, the Brazilian team came to Germany in order to get the needed knowledge directly from the customer. In this time, one-on-one relationships between the Brazilian team, the customer and project management in Germany were established. This facilitated collaboration later on because people knew each other in person and not only via email and telephone.\\
		11:43 -- 12:45&In larger deals that involve a larger team, such deep collaboration is usually not established. Instead, the working relationship is managed via \glspl{sla} and \glspl{kpi}, where quality and quantity of deliverables are defined.\\
		12:46 -- 13:57&Neither approach is clearly superior to the other (personal collaboration vs. management via \gls{sla})\\
		13:57 -- 15:38&He had an experience once with an Indian Team, where money was spent on bringing people to Germany to improve collaboration and quality. Few months later, these people ended up leaving the project to further their careers, because having worked abroad is an achievement that enables people to earn more in India. So the money spent on improving collaboration was essentially burned.\\
		15:38 -- 17:09&In the first three months, it is good to build personal relationships with team members. In the long run, there are two options. One option is the really deep personal exchange outlined in the example of the Brazilian team , which has the downside of increasing volatility in the team and is not a standard approach. The other option is to draw motivation out of the contract and out of being successful in fulfilling the contract. \\
		17:12 -- 19:01&Personal relationships are very important for employee satisfaction, but there are two possible identifications for people working offshore for a project: one is the identification with the project itself and being motivated by the local team lead. The other possibility is getting into the personal relationship with the customer (can be both T-Systems and the end customer) and identifying as part of a team.\\
		19:01 -- 19:25&Such identification with a global delivery team is not possible in large teams (50+ persons), in his experiences.\\
		19:25 -- 20:15& If the onsite and the offshore team share the same tasks (``Verlängerte Werkbank''), the team size is usually less than 30 people. The project manager is then distributing tasks directly to offshore team members.\\
		20:15 -- 20:36 &If the team is large enough to be organized into different organizational layers, e.g. local project managers or team leads, these personal relationships get lost.\\
		21:05 -- 22:22&There is the clich\'{e} that in the US, there is a certain motivation culture that involves a lot of enthusiasm, whereas in Germany, there is a lot of focus on the organization and the end result. Both have a certain truth to them but do not cover reality. Similarly, in general people are happier when working in an integrated way in an offshore team. The prerequisite is that the tasks enable this working mode.\\
		25:00 -- 26:47&In smaller scale collaborations, it is important to know the people you are working with on a personal level, not only by a name and picture. Especially in Munich, he has hosted so many offshoring partners that he is now one of the best tourist guides. He shows them the sights in order to let his guests learn about our cultural background and to start a discussion. This is helpful in building personal relationships.\\
		27:55 -- 28:50&Thanking the interview partner and finalization\\
		
\end{longtable}

\tocless\subsection{A.S. Viswanathan, 07/07/2016}
\label{int:Viswanathan}

\begin{longtable}{l p{12.5cm}}
	\textbf{Time} & \textbf{Summary} \\ 
	00:34 -- 02:54 & Introduction and consent to recording\\
	02:54 -- 04:24& Viswanathan is electrical engineer with a specialization in industrial engineering. In 1978, he started his career with English Electric which was a part of the General Electric Group. He worked there for two years, then he changed employers and started with Siemens. He held several positions, from the shop floor to CIO of the IT subsidiary of Siemens in India. Later, he moved on to the board of Siemens Information Systems, a software company that took global mandate within the Siemens Group.\\
	04:24 -- 06:00 & His responsibilities with Siemens were primarily the Business Solutions, as well as pioneering offshoring SAP with his team. Furthermore, he was responsible for IT services. In 2007, Siemens merged all local IT companies (mentioned are India, Germany, Austria, Switzerland, and Greece) into a new company called IT Services and Solutions. Viswanathan was on the executive management of this company and headed Global Portfolio of Mobility which included Transportation and Logistics on water, air etc.\\
	06:00 -- 07:00 & After taking a break in 2011, he founded his own management consultation company in 2012 with primarily customers from Germany, China and India.\\
	07:00 -- 08:20&When they conceptualized offering offshoring services the first customers were from the U.S. and the UK. They were very quick in understanding the cost advantages of offshoring and seizing the opportunity, not only shifting single tasks, but the entire operations to India. Customers in the U.S. were fairly open to offshoring. Viswanathan had the feeling that they would just go ahead and implement offshoring, so there were not many problems.\\
	08:20 -- 09:34&Working relationship with the U.S., in his experience, have been positive. Contributing to that was English being a common language and management meetings and schedules were easily set up. The \glspl{sla} were more critical. A reason for the positive experience could be that U.S. companies had virtually no plan B with respect to offshoring. When they did offshore, they took the time to evaluate different vendors, but once the decision was made, they just went with it and the work was shipped to India. This means there was a necessity to make the relationship work.\\
	09:34 -- 10:14& After the initial cycles of new contracts, the focus shifted to delivery. There would be a next wave of requirements with better productivity standards. In this phase, the facts and figures dominated working relationships, especially \glspl{sla}.\\
	10:14 -- 11:40&The American approach to offshoring is characterized by legalistic and contractual considerations. They always have consultants as part of the team who would do a good background search of the companies that provide operations in India. Then they would select three to four companies, travel to India and go to presentations of the chosen companies and then enter contract negotiations. They spent a lot of time on the contracting, so on the commercial \glspl{sla} and not so much on the processes. They believed it would be done and depended on that.\\
	11:40 -- 13:46&When it came to bringing Indian onsite teams to the U.S., for example for transitions, visa issues caused an unexpected amount of problems. This went so far that it became an administrative aspect of the discussion of offshoring with new customers.\\
	13:46 -- 15:03& The offshorability of an American job is very high. The work is conceptualized in a way that the needed skills can easily be managed. It is not a holistic job, but a specific set of tasks where a specific set of skills is needed. This goes back to U.S. companies already having experience in shifting tasks, but within the U.S.. People are working in different places and different time zones within the U.S., so they were already used to working in such a way.\\
	15:03 -- 16:39 & Jobs are very transaction-based in the U.S.. The education system facilitates that the single employee does not need to have an end-to-end knowledge of the entire process or the bigger picture of why this process works in that way. This is connected to the higher fluctuation of employees in American companies. Simultaneously, it is a huge advantage when it comes to offshoring. It would only need some kind of minimal training to get a new person for the job up and running.\\
	16:40 -- 18:24 & In German companies, the job design is more intrinsic and process-oriented. For example, a buyer in Germany compared to a buyer in the U.S. has a more specific background, maybe an apprenticeship in the field or some kind of special training. So when it comes to customizing an SAP system, a German system would differ greatly from an American system, because the role of an individual is more process- and system-oriented and holistic in Germany.\\
	18:24 -- 19:30& This role system presents a problem when shifting tasks offshore, because it can't be transferred like to like. That means, one person in Germany cannot simply be replaced with another person in India, because that person lacks the specific background and experience with the German company. Also, knowledge requirements are very intricate. In the buyer example, the employee needs knowledge in costing, the market, product design and so on. They wouldn't consult a technical specialist for those details. This system-oriented thinking is an advantage of the German society, but also an obstacle for offshoring.\\
	19:30 -- 19:55& Employees in Germany have proper education, vocational training and guidance and are very competent when they start the job. This competence is mirrored in the SAP systems and when the SAP system is then moved to India, there is a problem because there are no employees with the same skill set there.\\
	19:55 -- 21:00& Offshoring is a very alien concept to Germany, especially for \textit{Mittelstand} companies\footnote{\Acrlong{sme}}. Those grow very organically from family businesses, so they have in-built control of all processes. When it comes to outsourcing within Germany, the service provider does not have the full control of the process, they only provide some parts. But when it comes to offshoring, control must be given up by the customer, which is a very foreign concept for German companies.\\
	21:00 -- 22:28&Even when it comes to replacing domestic outsourcing with foreign outsourcing, the customer would complain about the offshore service provider, because the local service provider had a very good knowledge of their company. They were entirely dependent on the local service provider, almost like they were an extension of their own company to the extent that if there would be a larger incident, the provider's employees would come running even outside of their normal service hours. Of course, this can't be replicated with an offshore service provider.\\
	22:56 -- 24:13& Germans display a high degree of detailing, so the standards for the service providers tend to be very high. This has been one of the biggest barriers, and it is important to change the mindset regarding this. Also, Germans tend to prefer \gls{fdi} over foreign offshoring.\\
	24:13 -- 25:15& He did recently consult a German company with setting up \gls{fdi} and remarked to them that this practice is very typical for German companies. Frequently, the objective is recreating the own organization in India (``Mini-Germany in India'')\\
	25:15 -- 26:09&From a business solutions perspective, Germans would love to have the cost arbitrage, but they need to have the cost arbitrage the way the jobs are designed.\\
	26:09 -- 26:31&Another important point is language. Germany has become more international, but many companies still prefer to have their systems built and maintained completely in German. This is not essentially wrong, because many Indians are learning German as well, but it becomes a handicap. English is more easily learned.\\
	26:31 -- 28:28& Germans are more used to exporting then to importing, so they are not used to other countries selling to them or being better than them in the provision of services. Also, there's the issue of managing the transition with the loss of jobs or the decrease of job security when implementing offshoring. \textit{Betriebsrat}\footnote{Work council} and unions make this process slow and difficult, whereas it is much easier in the U.S.. \\
	28:28 -- 29:14&This needs to be accepted and accounted for in the planning of offshoring, both on the side of Germany and India. An American company would be much quicker in making the transition to offshoring, which is a disadvantage for German companies.\\
	29:14 -- 29:39&The reason for German companies to choose \gls{fdi} over foreign outsourcing is the inability to change the structure of jobs in a way that makes them easily offshorable (``slice and dice'').\\
	30:30 -- 33:09& A German company which wants to make the most of offshoring needs to think of entire functions in a process or in the company they would offshore, rather than offshoring just some minor roles. The second thing is, IT services can easily be offshored, meaning hardware, infrastructure support and similar tasks. But when it comes to process design, they can divide the job into a different level, so they can separate the decision-making part and the transaction part. Then, the transactions could be done elsewhere without many problems. But as long as processes are looked at in an integral way, there is a problem. So, the very deep level of job slice and dice that is common in America is not needed, but that one level of division could help German companies a lot.\\
	33:09 -- 33:33& This is one of the biggest issues, because at the moment one person makes the decisions but also posts data sets into SAP modules.\\
	33:33 -- 35:00 & With one customer, shortly before completing the implementation of offshoring they had to send back the work because unions had objected to the project. This was a very difficult scenario for both the service provider and the customer, as the setup in India was already completed.\\
	35:00 -- 36:48 & Application management was identified as one function that is easily offshorable. Change management with respect to other functions was more difficult, but application management could be dealt with by replacing one German resource with three Indian resources. This resulted in a much smaller cost arbitrage in an one-on-one comparison, but as multitasking was done in the Indian site, it yielded economic advantage. The distribution of work was here managed by Indian managers who had enough knowledge of the entire process. This is a lot more management effort than offshoring with American companies would entail.\\
	36:48 -- 38:49& Thanking the interview partner and finalization\\ 
\end{longtable}
\newpage
\tocless\subsection{Ingo Kümmritz}

\begin{longtable}{l p{12.5cm}}
	\textbf{Time} & \textbf{Summary} \\ 
	01:23 -- 02:16 & Introduction and consent to recording\\
	02:16 -- 03:35 & Ingo studied in the U.S. (High School and College). This has helped him to broaden his horizon when it comes to international delivery. In 2003, when he was working at IBM, the country manager approached him, asking if he could drive the topic of global delivery. After 10 years of working at IBM, he switched jobs and worked at Siemens. Via a short contract with an Indian company he ended up at NTT \nolinebreak Data in Germany.\\
	03:35 -- 04:50&He has been working in the area of global delivery for 13 years, mostly with India (90\%). In this time, he was responsible for big Projects and \gls{ams} deals. The critical point, regardless of the subject of the work, is communication and cooperation. It is necessary to learn how the counterpart on the service provider side is thinking and reacting to communication. So interaction between the partners is key, rather than processes, methods and tools.\\
	04:50 -- 06:08&Still, processes, methods and tools are the basis for any successful work, be it in Germany or in an international context. Twelve years back, he used to be in the role of a principal, which is a topic expert (as opposed to a people leader). Within Siemens, he moved on to a customer-facing role. At present, he holds a subject matter expert role again with NTT Data. So he has experience in delivery, sales, and customer-facing positions, which helps him understanding the partners involved.\\
	06:08 -- 07:21&On the question whether he was involved in offshoring from a customer point of view, he answers ``yes and no''. He explains that when talking about global delivery, there is no dedicated location for delivery, but rather a delivering company. This company then needs to find the right delivery model, concerning resources and locations. This is not necessarily India, but this country is the powerhouse in the industry. So he has been both on integrated delivery teams and on customer teams that travel to India to set up offshoring with a company there.\\
	07:20 -- 08:10 & It is clarified that his experience pertains to \gls{fdi}. Whenever he has worked with an Indian team, they were his colleagues at IBM or Siemens.\\
	08:22 -- 09:00&Although he did not have first-hand experience with offshoring in the U.S., he spent time there and had colleagues there, considering IBM is an American-based company. However, the German subsidiary would probably considered offshoring by the American headquarter, so in this way, he has experience in offshoring for a U.S. company.\\
	09:00 -- 09:57& India is a preferred partner when it comes to offshoring because the time zones are very convenient. Central European business hours can be covered from India without needing late shifts from the employees there.\\
	
\end{longtable}

\tocless\subsection{Subir Purkayastha}

\begin{longtable}{l p{12.5cm}}
	\textbf{Time} & \textbf{Summary} \\ 
	
\end{longtable}	

\end{appendix}	